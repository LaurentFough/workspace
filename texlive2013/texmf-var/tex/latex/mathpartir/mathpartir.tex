%  Mathpartir --- Math Paragraph for Typesetting Inference Rules
%
%  Copyright (C) 2001, 2002, 2003, 2005 Didier R�my
%
%  Author         : Didier Remy 
%  Version        : 1.2.1
%  Bug Reports    : to author
%  Web Site       : http://pauillac.inria.fr/~remy/latex/
% 
%  Mathpartir is free software; you can redistribute it and/or modify
%  it under the terms of the GNU General Public License as published by
%  the Free Software Foundation; either version 2, or (at your option)
%  any later version.
%  
%  Mathpartir is distributed in the hope that it will be useful,
%  but WITHOUT ANY WARRANTY; without even the implied warranty of
%  MERCHANTABILITY or FITNESS FOR A PARTICULAR PURPOSE.  See the
%  GNU General Public License for more details 
%  (http://pauillac.inria.fr/~remy/license/GPL).
%
%%%%%%%%%%%%%%%%%%%%%%%%%%%%%%%%%%%%%%%%%%%%%%%%%%%%%%%%%%%%%%%%%%%%%%%%%%%%
%  File mathpartir.tex (Documentation)
%%%%%%%%%%%%%%%%%%%%%%%%%%%%%%%%%%%%%%%%%%%%%%%%%%%%%%%%%%%%%%%%%%%%%%%%%%%%

\documentclass {article}

\usepackage {mathpartir}
\usepackage {listings}
\usepackage {array}
\usepackage {url}
\newif \ifhevea 
%HEVEA \heveatrue
\ifhevea 
\usepackage {hevea}
\fi

\lstset {basicstyle=\tt}
\let \lst \verb

\title {
%HEVEA\begin{tabular}{>{\huge}c}
\textbf{MATH} formulas in \\ 
\textbf{PAR}ragraph mode\\[-1ex]
\ifhevea\else{}\hskip 0em\hbox to 12em {\hrulefill}\\\fi
%HEVEA\hline
\textbf Typesetting \textbf Inference \textbf Rules \\
%HEVEA\end{tabular}
}
\author {Didier R{\'{e}}my}
\date {(Version 1.2.0, last modified 20/12/2005)}


\begin{document}

\maketitle

\begin{abstract}
This package provides macros for displaying lists of formulas that are
typeset in mixed horizontal and vertical modes. The package is two-folded.

The first part is an environment \verb"mathpar" that generalizes the math
display mode to allow several formulas on the same line, and several lines
in the same display. The arrangement of the sequence of formulas into lines
is automatic depending on the line width and on a minimum inter-formula
space and line width alike words in a paragraphs (in centerline mode).  A
typical application is  displaying a set of type inference rules.

The second par is a macro \lst"inferrule" to typeset inference rules
themselves.  Here again, both premises and conclusions are presented as list
of formulas that should be displayed in almost the same way, except that the
width is not fixed in advance; and the inference rule should use no more
width than necessary so that other inference rules are given a chance to
appear on the same line.

Although \verb"mathpar" and \verb"inferrule" look similar in their
specification, and are often used in combination, they are in fact
completely different in their implementations.
\end{abstract}

\ifhevea
 \section {License}

Mathpartir is Copyright (C) 2001, 2002 INRIA.  Mathpartir has been developed
by Didier R{\'{e}}my.  Mathpartir is free software; you can redistribute it
and/or modify it under the terms of the GNU General Public License as
published by the Free Software Foundation; either version 2, or (at your
option) any later version.  See the GNU General Public License for more
details (\url{http://pauillac.inria.fr/~remy/license/GPL}).  Mathpartir is
distributed in the hope that it will be useful, but without any
warranty. 
\fi

\section {The mathpar environment}

The mathpar environment is a ``paragraph mode for formulas''. 
It  allows to typeset long list of formulas putting as
many as possible on the same line: 
$$
\begin{tabular}{m{0.45\hsize}m{0.45\hsize}}
\begin{lstlisting}{Ocaml}
\begin{mathpar}
A-Formula \and 
Longer-Formula \and 
And \and The-Last-One
\end{mathpar}
\end{lstlisting}
&
\begin{mathpar}
A-Formula 
\and
Longer-Formula
\and
And 
\and
The-Last-One
\end{mathpar}
\end{tabular}
$$
Formulas are separated by \verb"\and" (or equivalently by a blank line). 
To enforce a vertical break it sufficies to replace \verb"\and" by
\verb"\\". 

The implementation of \verb"mathpar" entirely relies on the paragraph mode
for text. It starts  a new paragraph, and a math formula within a paragraph,
after adjusting the spacing and penalties for breaks. Then, it simply binds
\verb"\and" to something like \verb"\goodbreak". 

\section {The inferrule macro}

The inferrule macro is designed to typeset inference rules.  It should
only\footnote {Even though the basic version may work in text mode,
we discourage its use in text mode; the star-version cannot be used in
text-mode} be used in math mode (or display math mode). 

The basic use of the rule is 
\begin{verbatim}
\inferrule
  {one \\ two \\ three \\ or \\ more \\ premisses}
  {and \\ any \\ number \\ of \\ conclusions \\ as \\ well}
\end{verbatim}
This is the rendering on a large page
\def \one {\inferrule
  {one \\ two \\ three \\ or \\ more \\ premisses}
  {and \\ any \\ number \\ of \\ conclusions \\ as \\ well}
}
$$
\ifhevea \one
\else
\fbox {\vbox {\advance \hsize by -2\fboxsep \advance \hsize by -2\fboxrule
       \linewidth\hsize
        $$\one$$}}
\fi
$$
However, the same formula on a narrower page will automatically be typsetted
like that:
$$
\ifhevea \one
\else
\fbox {\hsize 0.33 \hsize \vbox {$$\one$$}}
\fi
$$
An inference rule is mainly composed of a premisse and a conclusion. 
The premisse and the conclusions are both list of formulas where the
elements are separated by \verb"\\". 

Note the dissymetry between typesetting of the premisses and of
conclusions where lines closer to the center are fit first. 

A newline can be forced by adding an empty line \verb"\\\\"

\begin{tabular}{m{0.44\hsize}m{0.44\hsize}}
\begin{lstlisting}{Ocaml}
\inferrule 
   {aa \\\\ bb}
   {dd \\ ee \\ ff}
\end{lstlisting}
&
$\inferrule {aa \\\\bb}{dd \\ ee \\ ff}$
\\
\end{tabular}

\subsection {Single rules}

Single rules are the default mode. 
Rules are aligned on their fraction bar, as illustrated below:
$$
\inferrule {aa \\ bb}{ee} \hspace {4em} \inferrule {aa \\\\ bb \\ ee}{ee}
$$
If the premise or the conclusion is empty, then the fraction bar is not
typeset and the premise or the conclusion is centered:
$$
\begin{tabular}{m{0.45\hsize}m{0.45\hsize}}
\begin{lstlisting}{Ocaml}
\inferrule {}{aa} + 
\inferrule {aa \\\\ aa}{}
\end{lstlisting}
&
$
\inferrule {}{aa} + 
\inferrule {aa \\\\ aa}{}
$
\\
\end{tabular}
$$
Use use \verb"{ }" instead of \verb"{}" to get an axiom for instance:
$$
\begin{tabular}{m{0.45\hsize}m{0.45\hsize}}
\begin{lstlisting}{Ocaml}
\inferrule { }{aa} + 
\inferrule {aa}{ }
\end{lstlisting}
&
\mbox {$
\inferrule { }{aa} + 
\inferrule {aa}{ }
$}
\\
\end{tabular}
$$

The macro \lst"\inferrule" acceps a label as optional argument, which will
be typeset on the top left corner of the rule: 
\par
\begin{tabular}{m{0.45\hsize}m{0.45\hsize}}
\begin{lstlisting}{Ocaml}
\inferrule [yop]
   {aa \\ bb}
   {cc}
\end{lstlisting}
&
$\inferrule [Yop]{aa \\ bb}{cc}$
\\
\end{tabular}
\par\noindent
See section~\ref {options} for changing typesetting of labels. 
A label can also be placed next to the rule directly, since the rule is
centered: 
\par
\begin{tabular}{m{0.45\hsize}m{0.45\hsize}}
\begin{lstlisting}{Ocaml}
\inferrule 
   {aa \\ bb}
   {cc}
\quad (\textsc {Yop})
\end{lstlisting}
&
$\inferrule{aa \\ bb}{cc} \quad (\textsc {Yop})$
\\
\end{tabular}

\subsection {Customizing presentation}

By default, lines are centerred in inference rules. 
However, this can be changed by either \lst"\mprset{flushleft}"
or \lst"\mprset{center}". For instance, 

\begin{tabular}{m{0.44\hsize}m{0.44\hsize}}
\begin{lstlisting}{Ocaml}
$$\mprset{flushleft}
  \inferrule 
    {a \\ bbb \\\\ ccc \\ dd}
    {dd \\ ee \\ ff}$$ 
\end{lstlisting}
&
$$\mprset{flushleft} 
\hsize 0.45\hsize
\inferrule {a \\ bb  \\ ccc \\ dddd}{e \\ ff \\ gg}$$
\\
\end{tabular}

\noindent
Note that lines are aligned independently in the premisse and the
conclusion, which are both themselves centered. In particular, 
left alignment will not affect a single-line premisse or conclusion. 

\subsection {Customizing rules}

One may wish to change use rules for rewriting rule or implications, etc.
There is a generic way of definition new rules by providing three parts: 
a tail, a body, and a head. The rule will then be built by joining
all three components in this order and filling the body with leaders to
extend as much as necessary. Here are examples

\begin{tabular}{m{0.54\hsize}m{0.44\hsize}}
\begin{lstlisting}{Ocaml}
$$\mprset{fraction={===}}
  \inferrule {a \\ bbb} {cc}$$ 
\end{lstlisting}
&
$$\mprset{fraction={===}}
  \inferrule {a \\ bbb} {cc}$$ 

\\
\begin{lstlisting}{Ocaml}
$$\mprset
  {fraction={\models=\Rightarrow}}
 \inferrule {a \\ bbb} {cc}$$ 
\end{lstlisting}
&
$$\mprset
  {fraction={\models=\Rightarrow}}
  \inferrule {a \\ bbb} {cc}$$ 
\\
\end{tabular}
The height and depth of the \emph{body} are used to adjust vertical space. 
One, may ``smash'' the body to reduce the vertical space

\begin{tabular}{m{0.54\hsize}m{0.44\hsize}}
\begin{lstlisting}{Ocaml}
$$\mprset
  {fraction={%
      {\scriptstyle\vdash}%
      {\smash-}%
      {\rightarrow\!\!}%
      }}
 \inferrule {a \\ bbb} {cc}\,\,$$ 
\end{lstlisting}
&
$$\mprset
  {fraction={{\scriptstyle\vdash}{\smash-}{\rightarrow\!\!}}}
  \inferrule {a \\ bbb} {cc}$$ 
\\
\begin{lstlisting}{Ocaml}
$$\mprset {fraction={\cdot\cdots\cdot}}
 \inferrule {a \\ bbb} {cc}$$ 
\end{lstlisting}
&
$$\mprset {fraction={{}{\,\smash\cdot\,}{}}}
  \inferrule {a \\ bbb} {cc}$$ 
\\
\end{tabular}
Since vertical skip does not take header and footer into account, which is
usually better but sometimes odd, this can be adjusted explicitly:

\begin{tabular}{m{0.54\hsize}m{0.44\hsize}}
\begin{lstlisting}{Ocaml}
$$\mprset
  {fraction={|=/},
   fractionaboveskip=0.6ex,
   fractionbelowskip=0.4ex}
 \inferrule 
   {a \\ bbb_{\downarrow}} 
   {cc^{\T\uparrow}}$$ 
\end{lstlisting}
&
$$\mprset
  {fraction={|=/},
   fractionaboveskip=0.6ex,
   fractionbelowskip=0.4ex}
 \inferrule 
   {a \\ bbb_{\downarrow}} 
   {cc^{\uparrow}}$$ 
\\
\end{tabular}
Finally, it is also possible to provide its own definition 
of fraction by

\begin{tabular}{m{0.54\hsize}m{0.44\hsize}}
\begin{lstlisting}{Ocaml}
\def \Over #1#2{\hbox{$#1 \over #2$}}
$$\mprset{myfraction=\Over}
  \inferrule {a \\ bbb} {cc}$$ 
\end{lstlisting}
&
\def \Over #1#2{\hbox{$#1 \over #2$}}
$$\mprset{myfraction=\Over}
  \inferrule {a \\ bbb} {cc}$$ 
\\
\end{tabular}

\paragraph{Customizing the horizontal skip between premisses}
(default value is 2em).
\begin{quote}
\begin{lstlisting}{Ocaml}
$$\mprset {sep=6em}
 \inferrule {a \\ bbb} {cc}$$ 
\end{lstlisting}
$$\mprset {sep=6em}
  \inferrule {a \\ bbb} {cc}$$ 
\end{quote}


\subsection {Tabulars in inference rules}

Although you probably do not want to do that, you may still use tabular
or minipages inside inference rules, but between braces, as follows:
$$
\begin{tabular}{m{0.50\hsize}m{0.50\hsize}}
\begin{lstlisting}{Ocaml}
\infer [Tabular-Rule]
{some \\ math \\ and \\
 {\begin{tabular}[b]{|l|r|}
 \hline Ugly & and
  \\[1ex]\hline 
  table & text
  \\\hline
 \end{tabular}} \\
 {\begin{minipage}[b]{6em}
  Do you really wish
  to do that? 
  \end{minipage}} \\ 
}
{some \\ conclusions}
\end{lstlisting}
&
\infer [Tabular-Rule]
{some \\ math \\ and \\
 {\begin{tabular}[b]{|l|r|}
 \hline Ugly & and
  \\[1ex]\hline
  table & text
  \\\hline
 \end{tabular}} \\
 {\begin{minipage}[b]{6em}
  Do you really wish
  to do that? 
  \end{minipage}} \\ 
}
{some \\ conclusions}
\\
\end{tabular}
$$


\subsection {Derivation trees}

To help writing cascades of rules forming a derivation tree, inference rules
can also be aligned on their bottom line. For this, we use the star-version:
$$
\begin{tabular}{m{0.65\hsize}m{0.45\hsize}}
\begin{lstlisting}{Ocaml}
\inferrule*
   {\inferrule* {aa \\ bb}{cc}
    \\ dd}
   {ee}
\end{lstlisting}
&
$
\inferrule*
   {\inferrule* {aa \\ bb}{cc}
    \\ dd}
   {ee}
$
\\
\end{tabular}
$$
The star version can also take an optional argument, 
but with a different semantics. The optional argument is parsed by the 
\verb"keyval" package, so as to offer a set of record-like options: 
$$
\def \val {{\em v}}
\def \arraystretch {1.4}
\begin{tabular}{|>{\tt}c|p{0.7\hsize}|}
\hline
\bf key & \bf Effect for value {\val}
\\\hline\hline
before & Execute {\val} before typesetting the rule.
         Useful for instance to change the maximal width of the rule.
\\\hline
width &  Set the width of the rule to {\val}
\\\hline
narrower & Set the width of the rule to {\val} times \verb"\hsize".
\\\hline
left & Put a label {\val} on the left of the rule
\\\hline
Left & Idem, but as if the label had zero width. 
\\\hline
Right & As \verb"Left", but on  the right of the rule.
\\\hline
right & As \verb"left",  but on the right of the rule.
\\\hline
leftskip & Cheat by (skip negative space) {\val} on the left side.
\\\hline
rightskip & Cheat by {\val} on the right side of the rule.
\\\hline
sep & Set the premise skip between rules to  {\val}. 
\\\hline
vdots & Raise the rule by {\val} and insert vertical dots. 
\\\hline
\end{tabular}
$$
Here is an example of a complex derivation:
$$
\inferrule* [left=Total]
  {\inferrule* [Left=Foo]
     {\inferrule* [Right=Bar,
                       leftskip=2em,rightskip=2em,vdots=1.5em]
         {a \\ a \\\\ bb \\ cc \\ dd}
         {ee}
       \\ ff \\ gg}
     {hh}
  \\
  \inferrule* [lab=XX,sep=4em]{uu \\ vv}{ww}}
  {(1)}
$$
and its code
\begin{lstlisting}{Ocaml}
\inferrule* [left=Total]
  {\inferrule* [Left=Foo]
     {\inferrule* [Right=Bar,
                       leftskip=2em,rightskip=2em,vdots=1.5em]
         {a \\ a \\\\ bb \\ cc \\ dd}
         {ee}
       \\ ff \\ gg}
     {hh}
  \\
  \inferrule* [lab=XX]{uu \\ vv}{ww}}
  {(1)}
\end{lstlisting}

\def \L#1{\lower 0.4ex \hbox {#1}}
\def \R#1{\raise 0.4ex \hbox {#1}}
\def \hevea {H\L{E}\R{V}\L{E}A}
\def \hevea {$\mbox {H}\!_{\mbox {E}}\!\mbox {V}\!_{\mbox {E}}\!\mbox {A}$}

\subsection {Implementation}

The main macro in the implementation of inference rules is the one that
either premises and conclusions.  The macros uses two box-registers one
\verb"hbox" for typesetting each line and one \verb"vbox" for collecting
lines. The premise appears as a list with
\verb"\\" as separator. Each element is considered in turn typeset in a
\verb"hbox" in display math mode. Its width is compare to the space left on
the current line. If the box would not fit, the current horizontal line is
transferred to the vertical box and emptied. Then, the current formula can
safely be added to the horizontal line (if it does not fit, nothing can be
done). When moved to the vertical list, lines are aligned on their center
(as if their left-part was a left overlapped). At the end the vbox is
readjusted on the right. 

This description works for conclusions. For premises, the elements must be
processes in reverse order and the vertical list is simply built upside
down. 

\section {Other Options}

\label {options}

The package also defines \verb"\infer" as a shortcut for \verb"\inferrule"
but only if it is not previously defined. 

The package uses \verb"\TirName" and \verb"\RefTirName" to typeset labels,
which can safely be redefined by the user. The former is used for defining
occurrences ({\em ie.} in rule \lst"\inferrule") while the latter is used
for referencing ({\em ie.} in the star-version). 

The vertical space in \verb"mathpar" is adjusted by
\verb"\MathparLineskip". To restore the normal paragraph parameters in mathpar
mode (for instance for some inner paragraph), use the command
\verb"\MathparNormalpar".  
The environment uses \verb"\MathparBindings" to
rebind \verb"\\", \verb"and", and \verb"\par". You can redefine thus command
to change the default bindings or add your own. 

\section {Examples}

See the source of this documentation ---the file \lst"mathpartir.tex"---
for full examples. 

\section {{\hevea} compatibility}

The package also redefines \verb"\hva" to do nothing in \lst"mathpar"
environment and nor in inference rules. 

In HeVeA, \verb"\and" will always produce a vertical break in mathpar
environment; to obtain a horizontal break, use \verb"\hva \and" instead.
Conversely, \verb"\\" will always produce a horizontal break in type
inference rules; to obtain a vertical break, use \verb"\hva \\" instead.

For instance, by default the following code,
\begin{lstlisting}{Ocaml}
\begin{mathpar}
\inferrule* [Left=Foo]
   {\inferrule* [Right=Bar,width=8em,
                 leftskip=2em,rightskip=2em,vdots=1.5em]
      {a \\ a \\ bb \\ cc \\ dd}
      {ee}
    \\ ff \\ gg}
   {hh}
\and
\inferrule* [lab=XX]{uu \\ vv}{ww}
\end{mathpar}
\end{lstlisting}
which typesets in {\TeX} as follows,
\begin{mathpar}
\inferrule* [Left=Foo]
   {\inferrule* [Right=Bar,width=8em,
                 leftskip=2em,rightskip=2em,vdots=1.5em]
      {a \\ a \\ bb \\ cc \\ dd}
      {ee}
    \\ ff \\ gg}
   {hh}
\and
\inferrule* [lab=XX]{uu \\ vv}{ww}
\end{mathpar}
would appear as follows with the compatible {\hevea} mode:
\begin{mathpar}
\inferrule* [left=Foo]
   {\inferrule* [right=Bar]
      {a \\ a \\ bb \\ cc \\ dd}
      {ee}
    \\ ff \\ gg}
   {hh}
\\
\inferrule* [lab=XX]{uu \\ vv}{ww}
\end{mathpar}
To obtain (almost) the same rendering as in {\TeX}, it could be typed as 
\begin{lstlisting}[escapechar=\%]{Ocaml}
\begin{mathpar}
\inferrule* [Left=Foo]
   {\inferrule* [Right=Bar,width=8em,
                 leftskip=2em,rightskip=2em,vdots=1.5em]
      {a \\ a \hva \\ bb \\ cc \\ dd}
      {ee}
    \\ ff \\ gg}
   {hh}
\hva \and
\inferrule* [lab=XX]{uu \\ vv}{ww}
\end{mathpar}
\end{lstlisting}
Actually, it would be typeset and follows with the compatible {\hevea} mode:
\begin{mathpar}
\inferrule* [left=Foo]
   {\inferrule* [right=Bar]
      {a \\ a \\\\ bb \\ cc \\ dd}
      {ee}
    \\ ff \\ gg}
   {hh}
\and
\inferrule* [lab=XX]{uu \\ vv}{ww}
\end{mathpar}


\end{document}

\end{document}

% LocalWords:  mathpar aa Yop bb dd ee ff cc Ocaml Foo leftskip rightskip vdots
% LocalWords:  gg hh uu vv ww HeVeA escapechar
 
