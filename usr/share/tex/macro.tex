% !TEX TS-program = latex
% !TEX encoding = Windows Latin 1
\documentclass[11pt]{article}

%\usepackage[backend=bibtex, citestyle=alphabetic, bibstyle=alphabetic, language=french]{biblatex}
%\addbibresource{../references.bib}

%Fourier for math | Utopia (scaled) for rm | Helvetica for ss | Latin Modern for tt
\usepackage[upright]{fourier} % math & rm
\usepackage[scaled=0.875]{helvet} % ss
\renewcommand{\ttdefault}{lmtt} %tt

% Latin Modern (similar to CM with more characters)
%\usepackage{lmodern} % math, rm, ss, tt

\usepackage{pdfsync}
\usepackage{geometry}
\geometry{a4paper}
\usepackage{amsmath}  % Maths
\usepackage{stmaryrd} % symboles bizarres (informatique)
\usepackage{listings} % Listings
\usepackage{moreverb} % Listings aussi
\usepackage{amsfonts,amssymb,verbatim} % ?
\usepackage{amsthm}   % Theor�mes
\usepackage{mathrsfs} % ?
\usepackage{graphicx} % m�me origine que color mais pour faire des graphs
\usepackage{pstricks,pst-plot,pstricks-add,pst-math,pst-node} % dessins
\usepackage[utf8]{inputenc} % encodage
\usepackage{color}		% couleurs
\usepackage[T1]{fontenc} % encodage police pdf
\usepackage{lastpage} % permet de se r�f�rer � la derni�re page d'un document
\usepackage[francais]{babel}
\usepackage{fancyhdr}
\usepackage{calc} 		% arithmetique infixe des compteurs
\newcommand{\noind}{\noindent}
\usepackage[bookmarks,colorlinks,breaklinks]{hyperref}  % PDF hyperlinks, with coloured links
\definecolor{dullmagenta}{rgb}{0.4,0,0.4}   % #660066
\definecolor{darkblue}{rgb}{0,0,0.4}
\hypersetup{linkcolor=blue,citecolor=blue,filecolor=dullmagenta,urlcolor=darkblue} % coloured links
%\hypersetup{linkcolor=black,citecolor=black,filecolor=black,urlcolor=black} % black links, for printed output

\newcommand{\imp}[1]{\textit{\textbf{#1}} }

 
\newcommand{\R}{\mathbb{R}}
\newcommand{\C}{\texttt{C}}
\newcommand{\Q}{\mathbb{Q}}
\newcommand{\Z}{\mathbb{Z}}
\newcommand{\N}{\mathbb{N}}
\newcommand{\K}{\mathbb{K}}

\newcommand{\ssi}{si et seulement si }
\newcommand{\re}{\mbox{Re} }
\newcommand{\im}{\mbox{Im} }
\newcommand{\vect}{\mbox{vect} }
\newcommand{\rg}{\mbox{rg} }
\newcommand{\Ker}{\mbox{Ker} }
\newcommand{\tr}{\mbox{tr} }
\newcommand{\Sp}{\mbox{S}_{p} }
\newcommand{\Quad}{\mbox{Quad} }
\newcommand{\grad}{\overrightarrow{\mbox{grad}} }
\newcommand{\rot}{\overrightarrow{\mbox{rot}} }

\newcommand{\noeud}{n\oe ud }
\newcommand{\noeuds}{n\oe uds }
\newcommand{\algo}{algorithme }
\newcommand{\algos}{algorithmes }

\newcommand{\IN}{\textit{in} }

\definecolor{gris}{rgb}{0.9,0.9,0.9} 
\definecolor{bleuclair}{rgb}{0.7,0.7,1}
\lstset{ basicstyle={\ttfamily \small}, language=c, keywordstyle=\color{blue}, frame=lines, backgroundcolor=\color{gris}, breaklines = true, showstringspaces=false} 

\newcommand{\afaire}{\begin{center}
\vspace{1cm}
{\Large \# \textit{IMAGE}}
\vspace{1cm}
\end{center}}

\newcommand{\abs}[1]{\left| #1 \right|}
\newcommand{\norme}[1]{\left\Vert #1 \right\Vert}
%\newcommand{\tribarre}[1]{\left\Vert \! \! \: \left| #1 \right| \! \!  \: \right\Vert}
%\newcommand{\dint}[1]{\displaystyle{\int} \! \!  \! \! \displaystyle{\int_{ #1}} }
%\newcommand{\tint}[1]{\displaystyle{\int} \! \!  \! \! \displaystyle{\int} \! \!  \! \! \displaystyle{\int_{#1}} }
\newcommand{\tribarre}[1]{\left\VERT #1 \right\VERT} % avec le pakage fourier
\newcommand{\dint}[1]{\displaystyle{\iint_{#1}}} % avec le pakage fourier
\newcommand{\tint}[1]{\displaystyle{\iiint_{#1}}} % avec le pakage fourier
\newcommand{\scal}[1]{\left\langle #1 \right\rangle}
\newcommand{\ent}[1]{\left\llbracket#1\right\rrbracket}
\newcommand{\somme}[2]{\displaystyle{\sum_{#1}^{#2}}}
\newcommand{\integrale}[2]{\displaystyle{\int_{#1}^{#2}}}
\newcommand{\e}{\mathrm{e}}
\def\jfrac#1#2{\raisebox{2pt}{$#1$}/\raisebox{-2pt}{$#2$}}   %%%%  jolie fraction


\pagestyle{fancy}
%\lhead{ } 
  %\chead{ }
% \rhead{ } 
%\fancyfoot{\cfoot{\thepage}}
\renewcommand{\headrulewidth}{0.6pt}
\renewcommand{\headrule}{{\color{gray}%
\hrule width\headwidth height\headrulewidth \vskip-\headrulewidth}}

\theoremstyle{plain} 
\newtheorem{theo}{Th\'eor\`eme } 
\newtheorem{rappel}{Rappel du th\'eor\`eme }
\newtheorem{lemme}{Lemme }

\theoremstyle{remark}
\newtheorem{remarque}[theo]{Remarque }

\theoremstyle{definition}
\newtheorem{d�finition}[theo]{D\'efinition }
